\documentclass{beamer}
\usepackage[utf8]{inputenc}
\usepackage{amsmath}
\usepackage{graphicx}

\usetheme{CambridgeUS}
\usecolortheme{whale} % Tema en azul

% Define colores personalizados
\definecolor{deepblue}{HTML}{003366}
\definecolor{lightblue}{HTML}{4682B4}
\definecolor{darktitle}{HTML}{5DADE2}
\definecolor{darkgreen}{HTML}{388E3C} % Color oscuro para títulos

% Configurar colores del frame title
\setbeamercolor{frametitle}{bg=darktitle,fg=white}
\setbeamercolor{frametitle continuation}{bg=darktitle,fg=white}

% Aplicar estilo al frame title
\setbeamertemplate{frametitle}[default][shadow=false]

% Aplica el azul oscuro a los puntos de las listas
\setbeamercolor{item}{fg=deepblue}

\title[Ing. Materiales]{Ingeniería de Materiales}
\subtitle{Ingeniería Industrial}
\author{ruben.velazquez@uteq.edu.mx}
\institute[UTEQ]{Universidad Tecnológica de Querétaro}
\date{Cuatrimestre Mayo - Agosto 2025}
\logo{\includegraphics[width=2cm]{../Imagenes/LogoUTEQ.png}} % Ajusta la ruta según donde tengas el logo

\makeatletter
\setbeamertemplate{footline}
{
	\leavevmode%
	\hbox{%
		\begin{beamercolorbox}[wd=.25\paperwidth,ht=2.25ex,dp=1ex,center]{author in head/foot}%
			\usebeamerfont{author in head/foot}\insertshortauthor
		\end{beamercolorbox}%
		\begin{beamercolorbox}[wd=.55\paperwidth,ht=2.25ex,dp=1ex,center]{title in head/foot}%
			\usebeamerfont{title in head/foot}\insertshorttitle
		\end{beamercolorbox}%
		\begin{beamercolorbox}[wd=.2\paperwidth,ht=2.25ex,dp=1ex,right]{section in head/foot}%
			\usebeamerfont{section in head/foot}\insertframenumber{} / \inserttotalframenumber\hspace*{2ex}
	\end{beamercolorbox}}%
	\vskip0pt%
}
\makeatother

\begin{document}

\frame{\titlepage}

\begin{frame}
    \frametitle{BIENVENIDA}
    \begin{itemize}
        \item \textbf{Asignatura:} Ingeniería de Materiales
        \item \textbf{Carrera:} Ingeniería Industrial en Competencias Profesionales
        \item \textbf{Modalidad:} Presencial asistida por tecnología
        \item \textbf{Período:} Mayo - Agosto 2025
        \item \textbf{Duración:} 45 horas totales (42 horas efectivas de 1 hora)
        \item \textbf{Plataforma:} Google Classroom
    \end{itemize}
\end{frame}

\begin{frame}[plain]
	\frametitle{¿Por qué son importantes los materiales?}
%	\begin{figure}
%		\centering
%		\includegraphics[width=.8\linewidth]{Imagenes/materiales_innovacion.jpg}
		%\caption{Variedad de materiales en aplicaciones industriales}
%	\end{figure}	
\textit{\begin{flushright}
		"Los materiales marcan la frontera de lo que es posible en ingeniería."
\end{flushright}}
	%% TODO: Puedes añadir notas aquí si es necesario
	\note{}	
\end{frame}
	
\begin{frame}
    \frametitle{Objetivos del Curso}
   \begin{block}{Objetivo Original (Hoja de Asignatura)}
   		El alumno utilizará los materiales de acuerdo al diseño del producto para que garantice la satisfacción del cliente y no contribuya al deterioro ambiental.
   \end{block}
   \begin{block}{Objetivo Propuesto}
   		El alumno seleccionará y utilizará materiales para aplicaciones industriales mediante el análisis de sus propiedades físicas, químicas y tecnológicas, aplicando criterios de funcionalidad, costo/desempeño e impacto ambiental, con apoyo de herramientas computacionales y de inteligencia artificial, para garantizar la satisfacción del cliente, la rentabilidad industrial y la minimización del impacto ambiental.
    \end{block}
\end{frame}

\begin{frame}
    \frametitle{Competencias a Desarrollar}
   
   \begin{itemize}
      \item Administrar los recursos necesarios de la organización para asegurar la producción planeada conforme a los requerimientos del cliente.
      
      \item Administrar el sistema de gestión de la calidad, con un enfoque sistémico, considerando factores técnicos y económicos, contribuyendo al desarrollo sustentable.
      
      \item Desarrollar e innovar sistemas de manufactura a través de la dirección de proyectos, considerando estándares de calidad, ergonomía, seguridad y ecología para lograr la competitividad y rentabilidad de la organización.
   \end{itemize}
   
\end{frame}

\begin{frame}
    \frametitle{Unidades Temáticas}
    Según Hoja de Asignatura:
    \begin{enumerate}
    	\item Propiedades de los materiales (15 horas)
    	\item Selección de materiales (30 horas)
    \end{enumerate}
    
    \vspace{0.5cm}
    
    \begin{alertblock}{Distribución de horas}
        \centering
        Total: 45 horas (18 teóricas, 27 prácticas)\\
        3 horas semanales
    \end{alertblock}
    
\end{frame}

\begin{frame}[plain]
	\frametitle{Mapa del Curso}
%	\begin{figure}
%		\centering
%		\includegraphics[width=1\linewidth]{Imagenes/mapa_materiales.png}
		%\caption{Mapa conceptual del curso}
%	\end{figure}	
	
	%% Si quieres añadir notas para usar durante la presentación
	\note{Explicar la relación entre las diferentes unidades y cómo se conectan con el proyecto integrador.}	
\end{frame}

\begin{frame}
	\frametitle{Políticas Institucionales}
	\begin{enumerate}
		\item Para tener derecho a asistir a clases y evaluaciones, es requisito estar inscritos oficialmente.
		\item Utilizaremos el correo institucional como medio oficial de comunicación.
		\item El plagio está estrictamente prohibido. Cualquier trabajo que no sea de su autoría resultará en la pérdida del derecho a aprobar la evaluación correspondiente.
		\item Se requiere un mínimo de 80\% de asistencia para tener derecho a evaluación.
		\item Las entregas deben ser puntuales y cumplir con los criterios establecidos.
		\item Las inasistencias solo pueden justificarse por causas específicas como enfermedad con incapacidad o solicitud de autoridad.
	\end{enumerate}

\end{frame}	

\begin{frame}
	\frametitle{Políticas Específicas del Curso}
	
	\textbf{Aspectos Académicos:}
	
	\begin{itemize}
		\item Entrega puntual de actividades con penalización por retraso
		\item Originalidad en todos los trabajos
		\item Uso ético y declarado de herramientas de IA
	\end{itemize}

	\textbf{Uso de Tecnología:}
	
	\begin{itemize}
		\item Aprovechamiento de bases de datos de materiales como apoyo
		\item Utilización de simulaciones para visualizar propiedades de materiales
		\item Enfoque en comprensión conceptual, no solo en resultados
	\end{itemize}
	
\end{frame}

\begin{frame}
    \frametitle{Metodología de Enseñanza-Aprendizaje}
    
    \textbf{Sesiones Presenciales}
    \begin{itemize}
        \item Exposición dialogada
        \item Análisis de casos industriales
        \item Prácticas de selección de materiales
        \item Discusión y trabajo colaborativo
    \end{itemize}
    \vspace{0.3cm}
    
    \textbf{Actividades Asincrónicas (Google Classroom)}
    \begin{itemize}
        \item Elaboración de fichas técnicas de materiales
        \item Resolución de problemas de selección
        \item Uso guiado de bases de datos y herramientas de IA
        \item Foros de discusión y análisis de casos
    \end{itemize}
\end{frame}

\begin{frame}
    \frametitle{Uso de Tecnología}
    
    \textbf{Bases de Datos de Materiales}
    \begin{itemize}
        \item MatWeb, CES EduPack o alternativas open source
        \item Recursos de repositorios académicos
    \end{itemize}
    \vspace{0.3cm}
    
    \textbf{Herramientas de IA como Apoyo}
    \begin{itemize}
        \item Asistentes conversacionales para:
        \begin{itemize}
            \item Profundizar en propiedades específicas
            \item Obtener explicaciones alternativas
            \item Verificar pasos en la selección de materiales
            \item Buscar información complementaria o aplicaciones industriales
        \end{itemize}
    \end{itemize}
\end{frame}
\begin{frame}
    \frametitle{Uso Ético de IA en el Curso}
    
    \begin{block}{Integración de IA en el proceso enseñanza-aprendizaje}
        Este curso incorpora el uso responsable de herramientas de Inteligencia Artificial como apoyo al aprendizaje, siguiendo lineamientos éticos institucionales.
    \end{block}
    
    \begin{columns}[t]
        \column{0.48\textwidth}
        \textbf{Principios rectores:}
        \begin{itemize}
            \item Integridad académica
            \item Transparencia
            \item Responsabilidad
            \item Desarrollo de pensamiento crítico
        \end{itemize}
        
        \column{0.48\textwidth}
        \textbf{Objetivo de integración:}
        \begin{itemize}
            \item Potenciar, no sustituir, el aprendizaje
            \item Desarrollar competencias digitales
            \item Preparar para entorno profesional actual
            \item Fomentar criterio profesional
        \end{itemize}
    \end{columns}
\end{frame}

\begin{frame}
    \frametitle{Usos Permitidos y No Permitidos de IA}
    
    \begin{columns}[t]
        \column{0.48\textwidth}
        \textbf{Usos recomendados:} {\color{darkgreen}\checkmark}
        \begin{itemize}
            \item Clarificar conceptos complejos
            \item Generar explicaciones alternativas
            \item Verificar procesos de resolución
            \item Buscar y sintetizar información
            \item Mejorar redacción técnica
            \item Generar ideas preliminares
        \end{itemize}
        
        \column{0.48\textwidth}
        \textbf{Usos no permitidos:} {\color{red}\textsf{×}}
        \begin{itemize}
            \item Presentar contenido de IA como propio
            \item Completar evaluaciones sin autorización
            \item Evadir el proceso de aprendizaje
            \item Falsificar datos experimentales
            \item Sustituir análisis personal en evaluaciones
        \end{itemize}
    \end{columns}
    
    \vspace{0.5cm}
    \begin{alertblock}{Transparencia}
        Se requiere declarar explícitamente cuándo y cómo se ha utilizado IA en los trabajos.
    \end{alertblock}
\end{frame}


\begin{frame}
    \frametitle{Evaluación}
    
    \textbf{Requisito de asistencia:}
    \begin{itemize}
        \item Mínimo 80\% para tener derecho a evaluación
    \end{itemize}
    \vspace{0.2cm}
    
    \textbf{Evaluación Diagnóstica (0\%)}
    \begin{itemize}
        \item Cuestionario inicial (hoy)
    \end{itemize}
    \vspace{0.2cm}
    
    \textbf{Evaluación Formativa (40\%)}
    \begin{itemize}
        \item Portafolio de fichas técnicas de materiales (10\%)
        \item Reportes de prácticas y análisis de casos (10\%)
        \item Ejercicios de aplicación y problemas (10\%)
        \item Participación en discusiones y foros (5\%)
        \item Autoevaluación/coevaluación (5\%)
    \end{itemize}
    \vspace{0.2cm}
    
    \textbf{Evaluación Sumativa (60\%)}
    \begin{itemize}
        \item Evaluación teórico-práctica Unidad I (15\%)
        \item Evaluación teórico-práctica Unidad II (15\%)
        \item Proyecto integrador de selección de materiales (30\%)
    \end{itemize}
\end{frame}

\begin{frame}
    \frametitle{Niveles de Desempeño}
    
    \textbf{SA (Satisfactorio)}
    \begin{itemize}
        \item Comprensión básica de propiedades de materiales y criterios de selección
        \item 80\% de actividades formativas con calidad aceptable
        \item Mínimo 70\% en evaluaciones sumativas
    \end{itemize}
    \vspace{0.2cm}
    
    \textbf{DE (Destacado)}
    \begin{itemize}
        \item Comprensión profunda de conceptos e interrelaciones
        \item 100\% de actividades con alta calidad
        \item Mínimo 85\% en evaluaciones sumativas
        \item Propuestas alternativas con justificación técnica
    \end{itemize}
    \vspace{0.2cm}
    
    \textbf{AU (Autónomo)}
    \begin{itemize}
        \item Pensamiento crítico avanzado en selección de materiales
        \item Soluciones originales a problemas complejos
        \item Mínimo 95\% en evaluaciones sumativas
        \item Integración de aspectos técnicos, económicos y ambientales con visión sistémica
    \end{itemize}
\end{frame}

\begin{frame}
    \frametitle{Recursos Principales}
    
    \textbf{Bibliografía Base}
    \begin{itemize}
        \item Askeland, D. (2005). \textit{Ciencias e Ingeniería de Materiales}. International Thomson Editores.
        \item Callister, W.D. (1997). \textit{Introducción a la Ciencia e Ingeniería de los Materiales}. Editorial Reverté.
        \item Smith, W. F. (2007). \textit{Fundamentos de la Ciencia e Ingeniería de Materiales}. McGraw-Hill Interamericana.
        \item Shackelford, J.F. (2005). \textit{Introducción a la Ciencia de Materiales para Ingenieros}. Pearson Alhambra.
    \end{itemize}
    \vspace{0.3cm}
    
    \textbf{Recursos Digitales}
    \begin{itemize}
        \item Repositorios CONACYT, IPN, UNAM
        \item Base de datos MatWeb (acceso institucional)
        \item Classroom: código de acceso [insertar código]
    \end{itemize}
\end{frame}

\begin{frame}
    \frametitle{Políticas de Clase Institucionales}
    \begin{enumerate}
        \item Para tener derecho a asistir a clases y a la evaluación del aprendizaje correspondiente, será requisito que los alumnos estén inscritos oficialmente.
        
        \item Los alumnos contarán con correo institucional, que será el medio oficial para la comunicación y entrega de reportes, trabajos o actividades asignadas en la plataforma de Google.
        
        \item El plagio de tareas, proyectos, presentaciones, evaluaciones o prácticas, queda estrictamente prohibido. El alumno que sea sorprendido entregando resultados que no sean de su autoría, perderá derecho a aprobar la evaluación correspondiente.
    \end{enumerate}
\end{frame}

\begin{frame}
    \frametitle{Políticas de Clase Institucionales}
    \begin{enumerate}\setcounter{enumi}{3}
        \item El alumno tendrá derecho a la evaluación del aprendizaje siempre y cuando cumpla con las actividades encomendadas y entregue en tiempo y forma los productos de aprendizaje señalados.
        
        \item La puntualidad y asistencia, así como las actitudes y valores son criterios para evaluar el saber ser y aprobar la unidad en la fase ordinaria. El porcentaje mínimo de asistencia será del 80\% del total de horas de la unidad.
        
        \item El estudiante podrá justificar alguna inasistencia solamente en caso de incapacidad por enfermedad o a solicitud de alguna autoridad educativa, familiar o empresa debido a alguna situación especial.
    \end{enumerate}
\end{frame}

\begin{frame}
    \frametitle{Importancia de la Ingeniería de Materiales}
    
    \begin{itemize}
        \item 60-70\% del costo final de muchos productos está en sus materiales
        \item La selección adecuada impacta directamente en:
        \begin{itemize}
            \item Calidad y desempeño del producto
            \item Rentabilidad del proceso productivo
            \item Sostenibilidad ambiental
            \item Competitividad de la organización
        \end{itemize}
        \item Influye en la selección de procesos, equipos y tecnologías necesarias
        \item Determina el impacto ambiental del ciclo de vida del producto
    \end{itemize}
    
    \begin{block}{Ejemplo de impacto económico}
        Se estima que los problemas relacionados con la selección incorrecta de materiales causan pérdidas de entre el 3\% y 5\% del PIB en países industrializados.
    \end{block}
\end{frame}

\begin{frame}
    \frametitle{Próximas Sesiones y Proyecto Integrador}
    
    \textbf{Sesiones Iniciales}
    \begin{itemize}
        \item Hoy: Presentación y diagnóstico
        \item Sesión 2: Introducción a la ingeniería de materiales
        \item Sesión 3: Interrelaciones entre estructura, propiedades y procesamiento
    \end{itemize}
    \vspace{0.2cm}
    
    \textbf{Proyecto Integrador}
    \begin{itemize}
        \item Selección y justificación de materiales para un producto industrial específico
        \item Consideración de aspectos técnicos, económicos y ambientales
        \item Aplicación de metodologías sistemáticas de selección
        \item Implementación gradual durante el cuatrimestre
        \item Presentación final en las últimas sesiones
    \end{itemize}
\end{frame}

\begin{frame}
    \frametitle{Evaluación Diagnóstica}
    
    A continuación realizaremos:
    \begin{enumerate}
        \item Presentación breve de cada estudiante
        \item Expectativas sobre el curso
        \item Cuestionario diagnóstico (Google Forms)
        \item Discusión sobre uso ético de bases de datos y herramientas de IA como apoyo al aprendizaje
    \end{enumerate}
\end{frame}

\begin{frame}
    \frametitle{¡Comencemos!}
    
    \textbf{Contacto:}
    \begin{itemize}
        \item Correo electrónico: [insertar correo]
        \item Horario de consulta: [insertar horario]
        \item Classroom: [insertar enlace]
    \end{itemize}
    \vspace{0.8cm}
    
    \begin{center}
        \Large ¿Preguntas?
    \end{center}
\end{frame}

\end{document}