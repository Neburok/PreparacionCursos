\documentclass{beamer}
\usepackage[utf8]{inputenc}
\usepackage{amsmath}
\usepackage{graphicx}
\usepackage{textcomp} % Para el símbolo de marca registrada si fuera necesario

\usetheme{AnnArbor}
\usecolortheme{spruce}

\setbeamertemplate{itemize items}[default]
\setbeamertemplate{enumerate items}[default]

% define a darker brown
\definecolor{uwbrown}{HTML}{662200}
% apply dark brown to the item bullet points
\setbeamercolor{item}{fg=uwbrown}

\title[Guía Ética IA Resumida]{Guía Ética para el Uso de IA en Ingeniería (Versión Resumida)}
\subtitle{Principios y Recomendaciones Clave}
\author{Elaborado por: Rubén Velázquez Hernández (UTEQ) \\ (Basado en el documento "GuiaEtica.pdf")}
\institute[UTEQ]{Universidad Tecnológica de Querétaro}
\date{16 de Abril de 2025 (Versión 1.0)}
\logo{\includegraphics[width=2cm]{../Imagenes/Logo_uteq}} % Ajusta la ruta si es necesario

\makeatletter
\setbeamertemplate{footline}
{
	\leavevmode%
	\hbox{%
		\begin{beamercolorbox}[wd=.33\paperwidth,ht=2.25ex,dp=1ex,center]{author in head/foot}%
			\usebeamerfont{author in head/foot}\insertshortauthor
		\end{beamercolorbox}%
		\begin{beamercolorbox}[wd=.34\paperwidth,ht=2.25ex,dp=1ex,center]{title in head/foot}%
			\usebeamerfont{title in head/foot}\insertshorttitle
		\end{beamercolorbox}%
		\begin{beamercolorbox}[wd=.33\paperwidth,ht=2.25ex,dp=1ex,right]{section in head/foot}%
			\usebeamerfont{section in head/foot}\insertframenumber{} / \inserttotalframenumber\hspace*{2ex}
	\end{beamercolorbox}}%
	\vskip0pt%
}
\makeatother

\begin{document}
	
	% Diapositiva 1: Título e Introducción
	\begin{frame}
		\titlepage
		\begin{block}{Propósito de la Guía [cite: 2]}
			Orientar a estudiantes y docentes hacia prácticas que aprovechen efectivamente las herramientas de IA como apoyo al proceso de enseñanza-aprendizaje, manteniendo la integridad académica, promoviendo la equidad y asegurando el desarrollo de competencias profesionales sólidas y pertinentes[cite: 2].
		\end{block}
	\end{frame}
	
	% Diapositiva 2: Principios Rectores Clave y Usos Recomendados
	\begin{frame}
		\frametitle{Principios Rectores Clave y Usos Recomendados}
		\begin{columns}[T] % Organizar en dos columnas
			\begin{column}{0.5\textwidth}
				\textbf{Principios Rectores Esenciales[cite: 3, 4, 5, 6, 7, 8]:}
				\begin{itemize}
					\item Integridad académica [cite: 3]
					\item Transparencia [cite: 4]
					\item Responsabilidad Estudiantil [cite: 5]
					\item Equidad y Acceso Justo [cite: 6]
					\item Privacidad de Datos [cite: 7]
					\item IA como complemento, no sustituto del razonamiento [cite: 8]
				\end{itemize}
			\end{column}
			\begin{column}{0.5\textwidth}
				\textbf{Usos Recomendados de la IA[cite: 11, 12]:}
				\begin{itemize}
					\item \textbf{Aprendizaje:} Clarificar conceptos, generar explicaciones alternativas, practicar resolución de problemas[cite: 11].
					\item \textbf{Investigación:} Búsqueda y síntesis de información, generación de ideas preliminares[cite: 11].
					\item \textbf{Productividad:} Mejora de redacción, corrección gramatical, generación de plantillas[cite: 12].
				\end{itemize}
			\end{column}
		\end{columns}
	\end{frame}
	
	% Diapositiva 3: Usos No Permitidos y Directrices de Responsabilidad
	\begin{frame}
		\frametitle{Usos No Permitidos y Directrices Clave de Responsabilidad}
		\textbf{Principales Usos No Permitidos[cite: 13]:}
		\begin{itemize}
			\item Presentar trabajo de IA como propio sin procesamiento crítico[cite: 13].
			\item Usar IA para evaluaciones en tiempo real sin autorización[cite: 13].
			\item Evadir el aprendizaje solicitando soluciones completas sin comprensión[cite: 13].
			\item Falsificar datos o resultados[cite: 13].
		\end{itemize}
		\vspace{0.5em} % Pequeño espacio vertical
		\textbf{Directrices para el Uso Responsable:}
		\begin{itemize}
			\item \textbf{Transparencia y Citación[cite: 15]:} Declarar uso, documentar proceso y citar herramienta. Diferenciar contribuciones.
			\item \textbf{Verificación y Validación[cite: 15]:} Contrastar con fuentes confiables, revisar críticamente datos y procedimientos.
		\end{itemize}
	\end{frame}
	
	% Diapositiva 4: Desarrollo de Competencias y Contextos de Uso
	\begin{frame}
		\frametitle{Desarrollo de Competencias y Contextos de Uso}
		\textbf{Desarrollo de Competencias con IA[cite: 16]:}
		\begin{itemize}
			\item Usar IA como andamiaje para comprensión superior, no para sustituir el razonamiento personal[cite: 16].
			\item Analizar críticamente respuestas de IA y practicar reformulación de problemas[cite: 16].
		\end{itemize}
		\vspace{0.5em}
		\textbf{Contextos Específicos (Resumen)[cite: 17]:}
		\begin{itemize}
			\item \textbf{Tareas/Proyectos:} Permitido para brainstorming e ideas iniciales, con desarrollo y revisión crítica propia.
			\item \textbf{Evaluaciones:} Generalmente no permitido en exámenes (salvo autorización). En proyectos, declarar su uso.
		\end{itemize}
	\end{frame}
	
	% Diapositiva 5: Reflexión Ética y Compromiso
	\begin{frame}
		\frametitle{Reflexión Ética Personal y Compromiso}
		\textbf{Preguntas Clave para la Reflexión Ética[cite: 26, 27, 28, 29, 30, 31, 32]:}
		\begin{itemize}
			\item ¿El uso de IA en esta tarea contribuye a mi aprendizaje o lo obstaculiza? [cite: 26]
			\item ¿Podré explicar y defender el trabajo resultante como reflejo de mi comprensión? [cite: 27]
			\item ¿Estoy siendo transparente sobre cómo he utilizado la IA y he verificado la información? [cite: 28]
			\item ¿Este uso es consistente con los objetivos educativos y respeta derechos de otros? [cite: 29, 31]
		\end{itemize}
		\vspace{0.5em}
		\textbf{Compromiso con la Ética en IA (Extracto)[cite: 34]:}
		\begin{itemize}
			\item Utilizar la IA para potenciar, no sustituir, el aprendizaje[cite: 34].
			\item Mantener honestidad académica y desarrollar criterio profesional[cite: 34].
			\item Contribuir a una cultura de uso ético y responsable de la IA[cite: 34].
		\end{itemize}
	\end{frame}
	
	% Diapositiva 6: Caso de Estudio Ilustrativo y Nota Final
	\begin{frame}
		\frametitle{Caso Ilustrativo y Conclusión}
		\framesubtitle{Ejemplo: Uso en Proyectos de Ingeniería [cite: 20]}
		\begin{block}{Uso Ético vs. Cuestionable [cite: 20]}
			\textbf{Positivo:} Estudiante usa IA para ideas preliminares, documenta, pero desarrolla cálculos y justificaciones por sí mismo[cite: 20]. \\
			\textbf{Cuestionable:} Estudiante solicita a IA un diseño completo y lo presenta como propio sin comprensión profunda[cite: 20].
		\end{block}
		\vspace{1em}
		\begin{alertblock}{Nota Final [cite: 35, 37, 38]}
			Esta guía es un documento vivo, sujeto a actualización[cite: 35, 37]. Fue desarrollada con asistencia de IA, pero revisada y aprobada por humanos, siguiendo los principios de esta misma guía[cite: 38].
		\end{alertblock}
		\vfill
		\centering
		Universidad Tecnológica de Querétaro
	\end{frame}
	
\end{document}