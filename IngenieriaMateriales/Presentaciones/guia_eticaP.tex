\documentclass{beamer}
\usepackage[utf8]{inputenc}
\usepackage{amsmath}
\usepackage{graphicx}
\usepackage{textcomp} % Para el símbolo de marca registrada si fuera necesario

\usetheme{AnnArbor}
\usecolortheme{spruce}

\setbeamertemplate{itemize items}[default]
\setbeamertemplate{enumerate items}[default]

% define a darker brown
\definecolor{uwbrown}{HTML}{662200}
% apply dark brown to the item bullet points
\setbeamercolor{item}{fg=uwbrown}

\title[Guía Ética IA - Ing.]{Guía Ética para el Uso de Inteligencia Artificial en Contextos Académicos de Ingeniería}
\subtitle{Principios y Recomendaciones}
\author{Elaborado por: Rubén Velázquez Hernández (UTEQ) \\ (Basado en el documento "GuiaEtica.pdf")}
\institute[UTEQ]{Universidad Tecnológica de Querétaro}
\date{16 de Abril de 2025 (Versión 1.0)}
\logo{\includegraphics[width=2cm]{../Imagenes/Logo_uteq}} % Ajusta la ruta si es necesario

\makeatletter
\setbeamertemplate{footline}
{
	\leavevmode%
	\hbox{%
		\begin{beamercolorbox}[wd=.33\paperwidth,ht=2.25ex,dp=1ex,center]{author in head/foot}% % Ajustado wd para mejor distribución
			\usebeamerfont{author in head/foot}\insertshortauthor
		\end{beamercolorbox}%
		\begin{beamercolorbox}[wd=.34\paperwidth,ht=2.25ex,dp=1ex,center]{title in head/foot}% % Ajustado wd
			\usebeamerfont{title in head/foot}\insertshorttitle
		\end{beamercolorbox}%
		\begin{beamercolorbox}[wd=.33\paperwidth,ht=2.25ex,dp=1ex,right]{section in head/foot}% % Ajustado wd
			\usebeamerfont{section in head/foot}\insertframenumber{} / \inserttotalframenumber\hspace*{2ex}
	\end{beamercolorbox}}%
	\vskip0pt%
}
\makeatother

\begin{document}
	
	% Diapositiva de Título
	\begin{frame}
		\titlepage
	\end{frame}
	
	% Diapositiva 2: Introducción
	\begin{frame}
		\frametitle{Presentación}
		\begin{itemize}
			\item Esta guía establece los principios éticos, responsabilidades y recomendaciones para el uso apropiado de tecnologías de Inteligencia Artificial (IA) en el contexto académico de las carreras de ingeniería.
			\item \textbf{Propósito:} Orientar a estudiantes y docentes hacia prácticas que aprovechen efectivamente las herramientas de IA como apoyo al proceso de enseñanza-aprendizaje, manteniendo la integridad académica, promoviendo la equidad y asegurando el desarrollo de competencias profesionales sólidas y pertinentes.
		\end{itemize}
	\end{frame}
	
	% Diapositiva 3: Fundamentos y Principios - Principios Rectores
	\begin{frame}
		\frametitle{1. Fundamentos y Principios}
		\framesubtitle{1.1 Principios Rectores}
		El uso de IA en contextos académicos de ingeniería debe guiarse por:
		\begin{itemize}
			\item \textbf{Integridad académica:} Priorizar el aprendizaje y el desarrollo de competencias profesionales.
			\item \textbf{Transparencia:} Comunicación clara sobre cuándo y cómo se utiliza la IA.
			\item \textbf{Responsabilidad:} Los estudiantes mantienen la responsabilidad final sobre su trabajo académico, así como su aprendizaje.
			\item \textbf{Equidad:} Acceso justo a las herramientas de IA y consideración de sus limitaciones.
			\item \textbf{Privacidad:} Protección de datos personales al interactuar con sistemas de IA.
			\item \textbf{Desarrollo de pensamiento crítico:} La IA como complemento, no como sustituto del razonamiento propio.
			\item \textbf{Sostenibilidad:} Consideración del impacto ambiental del uso intensivo de herramientas de IA.
		\end{itemize}
	\end{frame}
	
	% Diapositiva 4: Fundamentos y Principios - Marco Normativo
	\begin{frame}
		\frametitle{1. Fundamentos y Principios (Continuación)}
		\framesubtitle{1.2 Marco Normativo}
		Esta guía está alineada con:
		\begin{itemize}
			\item La Ley General de Educación Superior de México
			\item Los principios de la Nueva Escuela Mexicana
			\item Las recomendaciones de la UNESCO sobre Ética de la IA en Educación
			\item Las políticas institucionales sobre integridad académica de nuestra institución
		\end{itemize}
	\end{frame}
	
	% Diapositiva 5: Usos Permitidos y Recomendados de la IA - Herramienta de Aprendizaje
	\begin{frame}
		\frametitle{2. Usos Permitidos y Recomendados de la IA}
		\framesubtitle{2.1 Como Herramienta de Aprendizaje}
		Recomendado para:
		\begin{itemize}
			\item Utilizar IA para clarificar conceptos complejos de ingeniería.
			\item Generar explicaciones alternativas de temas estudiados en clase.
			\item Practicar resolución de problemas mediante verificación de procesos.
			\item Obtener retroalimentación sobre borradores de trabajos.
			\item Explorar aplicaciones prácticas de conceptos teóricos.
		\end{itemize}
	\end{frame}
	
	% Diapositiva 6: Usos Permitidos y Recomendados de la IA - Asistente de Investigación
	\begin{frame}
		\frametitle{2. Usos Permitidos y Recomendados de la IA (Continuación)}
		\framesubtitle{2.2 Como Asistente de Investigación}
		Recomendado para:
		\begin{itemize}
			\item Buscar y sintetizar información de fuentes confiables.
			\item Generar ideas preliminares para proyectos o investigaciones.
			\item Asistir en la revisión bibliográfica (con verificación posterior).
			\item Ayudar en la organización y estructuración de información.
			\item Traducir literatura técnica para acceder a conocimiento internacional.
		\end{itemize}
	\end{frame}
	
	% Diapositiva 7: Usos Permitidos y Recomendados de la IA - Herramienta de Productividad
	\begin{frame}
		\frametitle{2. Usos Permitidos y Recomendados de la IA (Continuación)}
		\framesubtitle{2.3 Como Herramienta de Productividad}
		Recomendado para:
		\begin{itemize}
			\item Mejorar la redacción técnica y académica.
			\item Asistir en la corrección gramatical y ortográfica.
			\item Generar plantillas o estructuras para documentos técnicos.
			\item Optimizar procesos repetitivos que no son el objetivo central del aprendizaje.
			\item Convertir datos entre formatos o sistemas de unidades.
		\end{itemize}
	\end{frame}
	
	% Diapositiva 8: Usos No Permitidos o Cuestionables - Sustitución del Aprendizaje
	\begin{frame}
		\frametitle{3. Usos No Permitidos o Cuestionables}
		\framesubtitle{3.1 Sustitución del Aprendizaje}
		No permitido:
		\begin{itemize}
			\item Presentar como propio el contenido generado íntegramente por IA sin procesamiento crítico.
			\item Utilizar IA para completar evaluaciones en tiempo real sin autorización.
			\item Evadir el proceso de aprendizaje solicitando soluciones completas sin comprenderlas.
			\item Falsificar datos experimentales o resultados de laboratorio.
			\item Sustituir el análisis personal por el generado por IA en actividades evaluativas.
		\end{itemize}
	\end{frame}
	
	% Diapositiva 9: Usos No Permitidos o Cuestionables - Prácticas Cuestionables
	\begin{frame}
		\frametitle{3. Usos No Permitidos o Cuestionables (Continuación)}
		\framesubtitle{3.2 Prácticas Cuestionables}
		Requiere consideración cuidadosa:
		\begin{itemize}
			\item Uso excesivo que limite el desarrollo de competencias fundamentales.
			\item Dependencia que reduzca la capacidad de trabajo autónomo.
			\item Aplicación acrítica de recomendaciones de IA sin verificación.
			\item Uso en áreas donde se requiere juicio profesional específico.
			\item Compartir información confidencial o datos sensibles con sistemas de IA.
		\end{itemize}
	\end{frame}
	
	% Diapositiva 10: Directrices para el Uso Responsable - Transparencia y Citación
	\begin{frame}
		\frametitle{4. Directrices para el Uso Responsable}
		\framesubtitle{4.1 Transparencia y Citación}
		\begin{itemize}
			\item \textbf{Declarar el uso:} Especificar claramente cuándo se ha utilizado IA como apoyo.
			\item \textbf{Documentar el proceso:} Explicar cómo se utilizó la IA (prompts, procesamiento posterior).
			\item \textbf{Citar adecuadamente:} Referenciar la herramienta de IA utilizada y fecha de consulta.
			\item \textbf{Diferenciar contribuciones:} Distinguir entre el contenido propio y el asistido por IA.
			\item \textbf{Ejemplo de citación:} \textit{"Este análisis fue desarrollado con asistencia de [Nombre de la herramienta de IA], consultada el [fecha]. Los prompts utilizados y el procesamiento posterior se describen en el Anexo X."}
		\end{itemize}
	\end{frame}
	
	% Diapositiva 11: Directrices para el Uso Responsable - Verificación y Validación
	\begin{frame}
		\frametitle{4. Directrices para el Uso Responsable (Continuación)}
		\framesubtitle{4.2 Verificación y Validación}
		\begin{itemize}
			\item Contrastar la información proporcionada por IA con fuentes académicas confiables.
			\item Revisar críticamente fórmulas, ecuaciones y procedimientos matemáticos.
			\item Verificar la precisión de datos técnicos y especificaciones.
			\item Evaluar la coherencia de las explicaciones con los principios ingenieriles establecidos.
			\item Identificar y corregir posibles sesgos en las respuestas generadas.
		\end{itemize}
	\end{frame}
	
	% Diapositiva 12: Directrices para el Uso Responsable - Desarrollo de Competencias
	\begin{frame}
		\frametitle{4. Directrices para el Uso Responsable (Continuación)}
		\framesubtitle{4.3 Desarrollo de Competencias}
		\begin{itemize}
			\item Utilizar la IA como andamiaje para alcanzar niveles superiores de comprensión.
			\item Complementar, no sustituir, el proceso de razonamiento personal.
			\item Analizar críticamente las respuestas generadas por IA.
			\item Practicar la reformulación de problemas para obtener mejores resultados.
			\item Desarrollar metacognición sobre el propio proceso de aprendizaje con IA.
		\end{itemize}
	\end{frame}
	
	% Diapositiva 13: Contextos Específicos de Uso - Tareas, Proyectos y Evaluaciones
	\begin{frame}
		\frametitle{5. Contextos Específicos de Uso}
		\framesubtitle{5.1 Tareas y Proyectos}
		\begin{itemize}
			\item \textbf{Brainstorming:} Permitido para generar ideas iniciales, con desarrollo posterior propio.
			\item \textbf{Borradores:} Aceptable para crear primeras versiones, con revisión crítica y reelaboración.
			\item \textbf{Revisión:} Recomendado para recibir retroalimentación, con implementación personal.
			\item \textbf{Investigación:} Útil para exploración inicial, requiere verificación y profundización.
		\end{itemize}
		\pause
		\framesubtitle{5.2 Evaluaciones}
		\begin{itemize}
			\item \textbf{Exámenes:} Generalmente no permitido, salvo autorización explícita.
			\item \textbf{Proyectos:} Uso permitido con transparencia y declaración apropiada.
			\item \textbf{Laboratorios:} Uso limitado a planificación y análisis posterior, no durante la ejecución.
			\item \textbf{Presentaciones:} Aceptable para estructuración y mejora, con dominio personal del contenido.
		\end{itemize}
	\end{frame}
	
	% Diapositiva 14: Contextos Específicos de Uso - Trabajo Colaborativo
	\begin{frame}
		\frametitle{5. Contextos Específicos de Uso (Continuación)}
		\framesubtitle{5.3 Trabajo Colaborativo}
		\begin{itemize}
			\item Establecer acuerdos claros sobre el uso de IA en equipos.
			\item Mantener equidad en las contribuciones personales.
			\item Utilizar IA para facilitar la integración de ideas diversas.
			\item Documentar colectivamente el uso de herramientas de IA.
		\end{itemize}
	\end{frame}
	
	% Diapositiva 15: Recomendaciones Prácticas - Elección de Herramientas y Formulación de Prompts
	\begin{frame}
		\frametitle{6. Recomendaciones Prácticas}
		\framesubtitle{6.1 Elección de Herramientas}
		\begin{itemize}
			\item Priorizar herramientas con transparencia sobre sus limitaciones.
			\item Considerar la especialización en temas de ingeniería.
			\item Evaluar las políticas de privacidad y uso de datos.
			\item Diversificar las herramientas para contrastar respuestas.
		\end{itemize}
		\pause
		\framesubtitle{6.2 Formulación Efectiva de Prompts}
		\begin{itemize}
			\item Ser específico sobre el contexto y nivel académico.
			\item Solicitar explicaciones de los procesos, no solo resultados.
			\item Pedir verificación de limitaciones o posibles errores.
			\item Solicitar referencias a fuentes confiables.
			\item Incluir instrucciones sobre formato y estilo técnico requerido.
		\end{itemize}
	\end{frame}
	
	% Diapositiva 16: Recomendaciones Prácticas - Evaluación Crítica de Respuestas
	\begin{frame}
		\frametitle{6. Recomendaciones Prácticas (Continuación)}
		\framesubtitle{6.3 Evaluación Crítica de Respuestas}
		\begin{itemize}
			\item Verificar la consistencia física y matemática.
			\item Contrastar con conocimientos previos y material del curso.
			\item Identificar simplificaciones excesivas o generalizaciones incorrectas.
			\item Reconocer cuando las respuestas parecen plausibles pero son incorrectas.
			\item Detectar respuestas evasivas o ambiguas.
		\end{itemize}
	\end{frame}
	
	% Diapositiva 17: Casos de Estudio y Ejemplos - Uso Ético en Proyectos de Ingeniería
	\begin{frame}
		\frametitle{7. Casos de Estudio y Ejemplos}
		\framesubtitle{7.1 Uso Ético en Proyectos de Ingeniería}
		\begin{block}{Caso positivo}
			Estudiante utiliza IA para generar ideas preliminares de diseño, documenta la inspiración, pero desarrolla cálculos, planos y justificaciones por sí mismo.
		\end{block}
		\begin{alertblock}{Caso cuestionable}
			Estudiante solicita a la IA que genere un diseño completo, incluidos cálculos y justificaciones, y lo presenta como trabajo propio sin comprensión profunda.
		\end{alertblock}
	\end{frame}
	
	% Diapositiva 18: Casos de Estudio y Ejemplos - Resolución de Problemas y Programación
	\begin{frame}
		\frametitle{7. Casos de Estudio y Ejemplos (Continuación)}
		\framesubtitle{7.2 Aplicación en Resolución de Problemas}
		\begin{block}{Uso recomendado}
			Resolver independientemente un problema, usar IA para verificar proceso y detectar errores, comprender las discrepancias.
		\end{block}
		\begin{alertblock}{Uso no recomendado}
			Solicitar directamente la solución sin intentar resolver el problema, copiar la respuesta sin analizar el proceso de resolución.
		\end{alertblock}
		\pause
		\framesubtitle{7.3 Apoyo en Programación y Simulación}
		\begin{block}{Enfoque ético}
			Utilizar IA para comprender conceptos de programación, mejorar algoritmos desarrollados personalmente y depurar errores específicos.
		\end{block}
		\begin{alertblock}{Enfoque cuestionable}
			Solicitar código completo para una tarea sin comprender su funcionamiento o sin capacidad para modificarlo y explicarlo.
		\end{alertblock}
	\end{frame}
	
	% Diapositiva 19: Preguntas para la Reflexión Ética (Parte 1)
	\begin{frame}
		\frametitle{8. Preguntas para la Reflexión Ética}
		Antes de utilizar herramientas de IA, considera:
		\begin{enumerate}
			\item ¿El uso de IA en esta tarea contribuye a mi aprendizaje o lo obstaculiza?
			\item ¿Podré explicar y defender el trabajo resultante como reflejo de mi comprensión?
			\item ¿Estoy siendo transparente sobre cómo he utilizado la IA?
			\item ¿He verificado críticamente la información proporcionada por la IA?
		\end{enumerate}
	\end{frame}
	
	% Diapositiva 20: Preguntas para la Reflexión Ética (Parte 2)
	\begin{frame}
		\frametitle{8. Preguntas para la Reflexión Ética (Continuación)}
		Antes de utilizar herramientas de IA, considera:
		\begin{enumerate}[resume] % Continúa la numeración de la lista anterior
			\item ¿Este uso de IA es consistente con los objetivos educativos de la actividad?
			\item ¿Estaría dispuesto a explicar a mi profesor cómo utilicé la IA en este trabajo?
			\item ¿El uso de IA en este contexto respeta los derechos de otros (privacidad, propiedad intelectual)?
			\item ¿Estoy desarrollando dependencia o fortaleciendo mi capacidad de criterio profesional?
		\end{enumerate}
	\end{frame}
	
	% Diapositiva 21: Recursos Adicionales y Compromiso con la Ética en IA
	\begin{frame}
		\frametitle{9. Recursos Adicionales y 10. Compromiso con la Ética en IA}
		\framesubtitle{9. Recursos Adicionales}
		\begin{itemize}
			\item Políticas institucionales: [Referencia a documentos específicos de la institución]
			\item Guías técnicas: Enlaces a recursos sobre uso efectivo de IA en ingeniería
			\item Herramientas de verificación: Recursos para contrastar información técnica
			\item Comunidades de práctica: Foros y grupos donde discutir aplicaciones éticas de IA
		\end{itemize}
		\pause
		\framesubtitle{10. Compromiso con la Ética en IA}
		Como miembro de la comunidad académica de ingeniería, me comprometo a:
		\begin{itemize}
			\item Utilizar la IA como herramienta para potenciar, no sustituir, mi aprendizaje.
			\item Mantener la honestidad académica.
			\item Desarrollar criterio profesional para evaluar críticamente los outputs de IA.
			\item Contribuir a una cultura de uso ético y responsable de la IA.
			\item Mantenerme actualizado sobre las implicaciones éticas.
			\item Priorizar el desarrollo de mis competencias fundamentales.
		\end{itemize}
	\end{frame}
	
	% Diapositiva 22: Nota Final
	\begin{frame}
		\frametitle{Nota Final}
		\begin{itemize}
			\item Esta guía está sujeta a actualización conforme evolucionen las tecnologías de IA y su aplicación en contextos educativos. Versión 1.0, [16 de Abril de 2025].
			\item \textbf{Nota sobre el desarrollo de este documento:} Este documento fue desarrollado con asistencia de IA (Claude 3.7 Sonnet, Anthropic), revisado, editado y aprobado por autores humanos, siguiendo los principios de transparencia y uso responsable de IA descritos en esta misma guía.
		\end{itemize}
		\vfill
		\centering
		Universidad Tecnológica de Querétaro \\ Facultad/Escuela de Ingeniería
	\end{frame}
	
\end{document}