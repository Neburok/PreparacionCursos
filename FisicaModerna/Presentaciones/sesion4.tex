\documentclass{beamer}
\usepackage[utf8]{inputenc}
\usepackage{amsmath}
\usepackage{graphicx}
\usepackage{hyperref}

\usetheme{AnnArbor}
\usecolortheme{spruce}

\setbeamertemplate{itemize items}[default]
\setbeamertemplate{enumerate items}[default]

% define a darker brown
\definecolor{uwbrown}{HTML}{662200}
% apply dark brown to the item bullet points
\setbeamercolor{item}{fg=uwbrown}

\title[Efecto Fotoeléctrico]{El Efecto Fotoeléctrico}
\subtitle{Fundamentos de la Física Cuántica}
\author{ruben.velazquez@uteq.edu.mx}
\institute[UTEQ]{Universidad Tecnológica de Querétaro}
\date{Cuatrimestre Mayo - Agosto 2025}
\logo{\includegraphics[width=2cm]{../Imagenes/Logo_uteq.png}}

\begin{document}
	
	\frame{\titlepage}
	
	\begin{frame}
		\frametitle{Objetivos de Aprendizaje}
		\begin{itemize}
			\item Explicar el fenómeno del efecto fotoeléctrico y su importancia para la física cuántica
			\item Comparar las predicciones clásicas con los resultados experimentales
			\item Aplicar la ecuación de Einstein para resolver problemas de efecto fotoeléctrico
			\item Relacionar el efecto fotoeléctrico con aplicaciones tecnológicas modernas
			\item Comprender cómo el efecto fotoeléctrico evidencia la naturaleza corpuscular de la luz
		\end{itemize}
	\end{frame}
	
	\begin{frame}
		\frametitle{Contexto Histórico}
		\begin{itemize}
			\item Descubierto por Heinrich Hertz (1887) mientras estudiaba ondas electromagnéticas
			\item Investigado sistemáticamente por Philipp Lenard (Premio Nobel 1905)
			\item Explicado teóricamente por Albert Einstein (1905) 
			\item Premio Nobel de Física a Einstein (1921) por "sus servicios a la Física Teórica y especialmente por su descubrimiento de la ley del efecto fotoeléctrico"
			\item Proporcionó evidencia crucial para la teoría cuántica naciente
		\end{itemize}
	\end{frame}
	
	\begin{frame}
		\frametitle{El Fenómeno del Efecto Fotoeléctrico}
		\begin{center}
			\includegraphics[height=5cm]{../Imagenes/EFE_diagrama}
		\end{center}
		\begin{itemize}
			\item Luz incidente sobre una superficie metálica provoca emisión de electrones
			\item Los electrones emitidos se denominan \alert{fotoelectrones}
			\item La energía cinética de los fotoelectrones puede medirse experimentalmente
		\end{itemize}
	\end{frame}
	
	\begin{frame}
		\frametitle{Montaje Experimental}
		\begin{columns}
			\column{0.5\textwidth}
			\includegraphics[width=\textwidth]{../Imagenes/EFE_diagrama}
			\column{0.5\textwidth}
			\begin{itemize}
				\item Tubo al vacío con dos placas metálicas
				\item Emisor (E): placa metálica expuesta a la luz
				\item Colector (C): recoge los electrones emitidos
				\item Batería: crea diferencia de potencial entre placas
				\item Amperímetro: mide la corriente de fotoelectrones
			\end{itemize}
		\end{columns}
	\end{frame}
	
	\begin{frame}
		\frametitle{Observaciones Experimentales (I)}
		\begin{columns}
			\column{0.5\textwidth}
			\includegraphics[width=\textwidth]{../Imagenes/EFE_curvas.png}
			\column{0.5\textwidth}
			\begin{itemize}
				\item Si $V > 0$: corriente aumenta hasta saturación
				\item Si $V < 0$: corriente disminuye
				\item Existe un \alert{potencial de frenado} ($V_s$) para el cual la corriente se anula
				\item La energía cinética máxima está relacionada con el potencial de frenado: $K_{max} = eV_s$
			\end{itemize}
		\end{columns}
	\end{frame}
	
	\begin{frame}
		\frametitle{Observaciones Experimentales (II)}
		
		\begin{enumerate}
			\item La energía cinética máxima de los fotoelectrones \alert{no depende de la intensidad} de la luz, sino de su \alert{frecuencia}
			\item Los electrones son emitidos instantáneamente (sin retraso), incluso a muy baja intensidad
			\item No hay emisión de electrones por debajo de cierta \alert{frecuencia de corte} específica para cada metal
			\item La emisión de fotoelectrones aumenta con la intensidad de la luz, pero no su energía máxima
		\end{enumerate}
		
	\end{frame}
	
	
	
	\begin{frame}
		\frametitle{Explicación de Einstein (1905)}
		\begin{itemize}
			\item La luz está compuesta por "cuantos" de energía (fotones)
			\item Cada fotón tiene una energía $E = hf$, donde $h$ es la constante de Planck
			\item Un fotón transfiere toda su energía a un solo electrón
			\item Parte de esta energía se usa para liberar al electrón del metal (función trabajo $\phi$)
			\item El resto se convierte en energía cinética del electrón
		\end{itemize}
		
		\begin{alertblock}{Ecuación de Einstein para el Efecto Fotoeléctrico}
			\begin{equation}
				K_{max} = hf - \phi
			\end{equation}
		\end{alertblock}
	\end{frame}
	
	\begin{frame}
		\frametitle{Interpretación de la Ecuación de Einstein}
		\begin{columns}
			\column{0.5\textwidth}
			\includegraphics[width=\textwidth]{../Imagenes/grafica_pendiente}
			
			\column{0.5\textwidth}
			\begin{itemize}
				\item $K_{max} = hf - \phi$
				\item $h$ = pendiente de la recta
				\item $\phi$ = función trabajo (intercepto)
				\item Frecuencia de corte: $f_c = \frac{\phi}{h}$
				\item Longitud de onda de corte: $\lambda_c = \frac{hc}{\phi}$
			\end{itemize}
		\end{columns}
	\end{frame}
	
	\begin{frame}
		\frametitle{Función Trabajo para Diversos Metales}
		\begin{center}
			\begin{tabular}{|c|c|c|}
				\hline
				\textbf{Metal} & \textbf{Función Trabajo $\phi$ (eV)} & \textbf{$\lambda_c$ (nm)}\\
				\hline
				Na & 2.46 & 504 \\
				Al & 4.08 & 304 \\
				Cu & 4.70 & 264 \\
				Zn & 4.31 & 288 \\
				Ag & 4.73 & 262 \\
				Pt & 6.35 & 195 \\
				Fe & 4.50 & 276 \\
				\hline
			\end{tabular}
		\end{center}
		
		\begin{block}{Nota}
			$hc \approx 1240 \text{ eV} \cdot \text{nm}$ es una combinación útil para cálculos
		\end{block}
	\end{frame}
	
	\begin{frame}
		\frametitle{Explicación de las Observaciones Experimentales}
		\begin{enumerate}
			\item $K_{max} = hf - \phi$ depende de la frecuencia $f$, no de la intensidad
			\item La emisión es instantánea porque cada fotón interactúa individualmente
			\item No hay efecto por debajo de $f_c$ porque $hf < \phi$
			\item Mayor intensidad = más fotones = más electrones, pero con la misma energía máxima
		\end{enumerate}
		
		\begin{alertblock}{¡La naturaleza corpuscular de la luz!}
			El efecto fotoeléctrico demuestra que la luz se comporta como partículas (fotones) al interactuar con la materia, no solo como ondas.
		\end{alertblock}
	\end{frame}
	
	\begin{frame}
		\frametitle{Ejemplo: Cálculo con Efecto Fotoeléctrico}
		\small
		\begin{block}{Problema}
			Una superficie de sodio (Na) con función trabajo $\phi = 2.46$ eV se ilumina con luz de longitud de onda de 300 nm. Determine:
			\begin{enumerate}
				\item La energía de los fotones incidentes
				\item La energía cinética máxima de los fotoelectrones
				\item El potencial de frenado
			\end{enumerate}
		\end{block}
		
		\begin{exampleblock}{Solución}
			\begin{align}
				E &= hf = \frac{hc}{\lambda} = \frac{1240 \text{ eV}\cdot\text{nm}}{300 \text{ nm}} = 4.13 \text{ eV}\\
				K_{max} &= hf - \phi = 4.13 \text{ eV} - 2.46 \text{ eV} = 1.67 \text{ eV}\\
				V_s &= \frac{K_{max}}{e} = 1.67 \text{ V}
			\end{align}
		\end{exampleblock}
	\end{frame}
	
	\begin{frame}
		\frametitle{Actividad con Simulador PhET}
		\begin{center}
			\includegraphics[height=6cm]{../Imagenes/grafica_pendiente}
		\end{center}
		\begin{block}{Instrucciones}
			Acceder a: {https://phet.colorado.edu/es/simulation/photoelectric}
		\end{block}
	\end{frame}
	
	\begin{frame}
		\frametitle{Aplicaciones Tecnológicas}
		\begin{columns}
			\column{0.6\textwidth}
			\begin{itemize}
				\item Celdas solares fotovoltaicas
				\item Sensores y detectores de luz
				\item Tubos fotomultiplicadores
				\item Dispositivos de acoplamiento de carga (CCD)
				\item Microscopía de fotoemisión
				\item Visión nocturna
				\item Lectores ópticos y controles automáticos
			\end{itemize}
			
			\column{0.4\textwidth}
			\includegraphics[width=\textwidth]{../Imagenes/grafica_pendiente}
		\end{columns}
	\end{frame}
	
	\begin{frame}
		\frametitle{Impacto Histórico y Conceptual}
		\begin{itemize}
			\item Primera evidencia experimental directa de la cuantización de la energía
			\item Confirmó la hipótesis cuántica de Planck sobre la radiación electromagnética
			\item Introdujo el concepto del fotón (partícula de luz)
			\item Contribuyó al desarrollo del principio de dualidad onda-partícula
			\item Pieza fundamental en la revolución cuántica del siglo XX
			\item Transformó nuestra comprensión de la interacción luz-materia
		\end{itemize}
	\end{frame}
	
	\begin{frame}
		\frametitle{Recapitulación: Conceptos Clave}
		\begin{itemize}
			\item El efecto fotoeléctrico es la emisión de electrones cuando la luz incide sobre un metal
			\item No puede explicarse mediante la física clásica (modelo ondulatorio)
			\item Requiere el modelo corpuscular de la luz (fotones con energía $E = hf$)
			\item La ecuación de Einstein: $K_{max} = hf - \phi$ describe correctamente el fenómeno
			\item Estableció las bases para el desarrollo de la mecánica cuántica
			\item Tiene numerosas aplicaciones tecnológicas en la actualidad
		\end{itemize}
	\end{frame}
	
	\begin{frame}
		\frametitle{Conexión con el Siguiente Tema}
		\begin{alertblock}{Dualidad Onda-Partícula}
			\centering
			Si la luz puede comportarse como partícula...\\
			¿Podría la materia comportarse como onda?
		\end{alertblock}
		
		\begin{center}
			\Large
			Hipótesis de De Broglie (1924):\\
			$\lambda = \frac{h}{p} = \frac{h}{mv}$
		\end{center}
	\end{frame}
	
	\begin{frame}
		\frametitle{Actividades y Evaluación}
		\begin{block}{Actividades para la próxima sesión}
			\begin{itemize}
				\item Resolver el problema integrador sobre aplicaciones del efecto fotoeléctrico
				\item Leer el material sobre la Hipótesis de De Broglie y responder guía de lectura
				\item Opcional (puntos extra): Video demostrativo sobre efecto fotoeléctrico
			\end{itemize}
		\end{block}
		
		\begin{alertblock}{Evaluación formativa}
			\begin{itemize}
				\item Quiz digital: acceso a través del código QR o enlace en Google Classroom
				\item Reporte de la actividad con el simulador PhET
				\item Participación en las discusiones de clase
			\end{itemize}
		\end{alertblock}
	\end{frame}
	
	\begin{frame}
		\frametitle{Referencias y Recursos Complementarios}
		\begin{thebibliography}{99}
			\bibitem{Griffiths} Griffiths, D. (2016). \textit{Quantum Mechanics}. Cambridge University Press.
			\bibitem{Eisberg} Eisberg, R. \& Resnick, R. \textit{Física Cuántica}. Limusa Wiley.
			\bibitem{Serway} Serway, R. A., Moses, C. J., \& Moyer, C. A. (2005). \textit{Física Moderna}.
			\bibitem{Tipler} Tipler, P. A. (2012). \textit{Física Moderna}. Reverté.
		\end{thebibliography}
		
		\begin{block}{Videos recomendados}
			\begin{itemize}
				\item "El efecto fotoeléctrico" - Walter Lewin, MIT OpenCourseWare
				\item "Explicación del efecto fotoeléctrico" - Khan Academy (español)
				\item "Aplicaciones modernas del efecto fotoeléctrico" - SciShow
			\end{itemize}
		\end{block}
	\end{frame}
	
	\begin{frame}
		\frametitle{¿Preguntas?}
		\begin{center}
			\huge ¡Gracias por su atención!\\
			\vspace{1cm}
			\normalsize
			Para dudas adicionales: [correo electrónico del profesor]
		\end{center}
	\end{frame}
	
\end{document}